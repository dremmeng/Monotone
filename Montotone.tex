\documentclass[10pt, oneside]{article} 
\usepackage{amsmath, amsthm, amssymb, calrsfs, wasysym, verbatim, bbm, color, graphics, geometry}

\geometry{tmargin=.75in, bmargin=.75in, lmargin=.75in, rmargin = .75in}  

\newcommand{\R}{\mathbb{R}}
\newcommand{\C}{\mathbb{C}}
\newcommand{\Z}{\mathbb{Z}}
\newcommand{\N}{\mathbb{N}}
\newcommand{\Q}{\mathbb{Q}}
\newcommand{\Cdot}{\boldsymbol{\cdot}}

\newtheorem{thm}{Theorem}
\newtheorem{defn}{Definition}
\newtheorem{conv}{Convention}
\newtheorem{rem}{Remark}
\newtheorem{lem}{Lemma}
\newtheorem{cor}{Corollary}
\newtheorem{example}{Example}
\newtheorem{exe}{Exercise}

\title{Monotone Groups}
\author{[Drew Remmenga]}


\begin{document}

\maketitle

\vspace{.25in}
\begin{abstract}
    This is an atempt to underpin the fundamental structure of infinite groups much as has been don efor finite groups. In this paper we outline a definition for classifying groups according to their structure. Infinite groups are more difficult to pin and place than finite groups. However we believe this paper has successfully done so and we term this new classification of groups monotone. 
\end{abstract}
\begin{defn}
    A group is called monotone if it is isomorphic to $\Z_{p_{1}^{\alpha_{1}}} \times ... \times \Z_{p_{n}^{\alpha_{n}}} \times \Z^{m} \times \R_{+}^{q} \times \R_{\neq 0}^{s}$
\end{defn}

\end{document}